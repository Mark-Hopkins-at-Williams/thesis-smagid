%% Introduction
\chapter{Introduction}
\label{chap:introduction}

\section{Motivation}

According to Karen Cheng's 2006 book \textit{Designing Type,} there were, at the time of publishing, likely over 300,000 individual typefaces available to users \cite{cheng2006}. Today, almost twenty years later, that number is no doubt higher; and yet, we use essentially the same font selection tools developed decades ago: a scrollable list of font names. All major word processors (Microsoft Word, Apple Pages, Google Docs) and most other graphic design tools use this decades-old interface. A few improvements have been made on the basic list-based font selection model (e.g. displaying font names in their own typeface and list alphabetization), but the fundamental aspects of the interface have remained unchanged. Here there arises an issue: compared with these early word processors, users have access to several orders of magnitude more typefaces---hundreds of thousands compared to dozens. Users cannot be expected to navigate through such a large number of fonts; scrolling through 300,000 items is not a reasonable ask. Moreover, this list-based interface does not align with the typical needs of a user: when searching for fonts, users often have a particular style in mind (professional, casual, festive), and this basic list-based interface does not incorporate any notion of style as part of its search. Notably, alphabetical order in a list of typefaces does not help a user who does not already know the name of the font they are searching for. After over forty years of little development in the area of font selection interfaces---while the numbers of available fonts has increased substantially---there is a need for better font selection tools, particularly interfaces which take into account meaningful aspects of typeface style.

\section{Goals and Contributions}

This thesis aims to address this problem. We show that training autoencoder-like neural networks on font image data yields meaningful style encoding vectors which quantify different aspects of typeface style, upon which useful style-based font selection tools can be built. We implement three neural network models, all variants of an autoencoder, which encode typeface style in the intermediate vector representation between the autoencoder's encoding and decoding blocks. Using the style encodings from the most effective model, we build a novel font selection webapp that leverages spatial information from the model. Evaluating our models' encoding spaces using attribute categories provided by Google Fonts, we find that our models effectively encode certain aspects of style, with similarity scores within style categories increasing as model sophistication increases. We additionally conducted a font-matching user study comparing our font selection tool against two baselines; while we are not able to draw conclusions about quantitative performance between tools given the small size of our user study, the qualitative feedback we collected suggests that our font selector webapp provided several desireable aspects for font exploration.

Ultimately, we find that the models we have implemented, especially our final model adapted from Srivatsan et al. \cite{srivatsan2020} and trained on our full dataset, effectively encode many aspects of typeface style. We additionally demonstrate---based on the preliminary findings from our user study---that these style encodings can be used to create useful style-based font selection interfaces. This research contributes to the field of typeface style inference and suggests that further work can be done towards the issue of creating effective and simple style-based font selection tools.

\section{Roadmap}

This thesis document is organized as follows. Chapter 2 explores some of the background and history of font selection tools, looking at early font selection interfaces as well as a couple of recently developed language-based tools; explains the structure of autoencoders and autoencoder-like models; and reviews related work on font selection and font inference. Chapter 3 details our three models (Basic Autoencoder, Style Transfer, and our model adapted from Srivatsan et al.), explains how we obtain the style encodings used by our font selection tool, discusses the design choices for our novel font selector interface, and details the end-to-end system upon which the tool is built. Chapter 4 includes a quantitative evaluation of font distance similarity based on attribute categories from the Google Fonts library, as well as a user evaluation of our font selection tool. Chapter 5 summarizes our findings, enumerates a few areas for future work, and explains some of the lessons learned in the process of this extended research project.